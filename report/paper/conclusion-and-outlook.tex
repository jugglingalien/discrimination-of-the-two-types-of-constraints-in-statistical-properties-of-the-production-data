\chapter{Conclusion and Outlook}
%\section{Thesis Contribution}
%\section{Outlook}

We linked these two data processing schemes: Fixed Step Sized binning and Fixed Bucket Sized binning, to different constraint categories and obtained different results in the steel production events analysis. Those results showed us how the association networks' topological features could hint at the constraints of a steel production system. However, the study of links between intrinsic constraints in a production system and some patterns we can observe in the production data is much larger than what started as an investigation in this thesis work.

The optimisation scheme integrated theoretical framework let us conduct simulation experiments. With Flux Balance Analysis, we constrained the system at different levels and allowed for fluctuations in the input and controlled the products via objective functions. 

Our abstract theoretical framework is a convenient starting point to conduct perturbation experiments for the existing production networks. On a further step, it should be structured with a random graph considering some additional consistency constraints instead of the one we picked from homo sapiens metabolism in this thesis work. In that random graph, one needs to make sure that the cycles in the graph are suitable to create reactions out of nothing, and some mass balance constraints need to be incorporated. 

The main challenge would be to construct arbitrary networks within the Operations Research framework. Therefore, different categories of constraints such as technical constraints, logistic constraints, physical and chemical constraints, economical constraints, and performance-indicator based constraints need to be quantified carefully in this framework.

%Possible hypotheses can be listed out, and then ask what tools would help us figure those out. We need to start from the production system. We are interested in constraints in the system. Those are the technological constraints (about the network) and the economical constraints.  Alternative paths in the network are a kind of reduction of constraints. Since the technological constraints can only be influences by the network architecture, having a vulnerable network is a technological constraint. Sensitivity to input fluctuations is similarly a technological constraint. There is also an economical constraint of the product portfolio. What difference in terms of output/production patterns (association networks derived from the production data) comes from one type of constraint and another type of constraint.