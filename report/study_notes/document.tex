"Production Constraints" which is a critical subject to optimization researches
arise from those technology-driven production types related to facility
capabilities.
Considering the roughly explained production steps above, each process has
its own technical or physical constraints.
Optimizing individual sequences in each process is a necessity for a successful
local process. Local constraints on different processes are integral parts of a
global optimization problem that is tackled by human expert planners for a
solution.
To what degree can above discussed local constraints be determinant
on the features of production orders that are taken into account by human
expert planners while they design a sequence that can be produced with a
minimum cost?
The technical and physical constraints mentioned in the introduction section
are obviously unique to those specific production lines and they vary under
differently customized production lines. Produced order properties such as
thickness, width, temperature, and chemical composition are the characteristic
features of order products that are possibly shaped under the effect of those
constraints.
production sequences contain production orders
with specific characteristics for the sake of an optimized production process
under several constraints effects.

Alternative Binning Methods for Network Nodes
Except for the Steel Grade feature network results, Width and Thickness feature
networks are quite dense and look random. Since width and thickness
characteristics for production orders might vary in large ranges, grouping orders
with similar values for the relevant features domain values will be an
efficient way to obtain meaningful network graph representations.

Therefore, applying alternative binning methods to Width and Thickness network
nodes might be handy to gather the production orders with tiny differences
together in common nodes and to gain more meaningful graph representations
to evaluate. Binning of the feature values can be performed in various
ways. Throughout this section, two ways of binning given below were applied
and investigated for the Width and Thickness features:
1. Keeping the bucket size (orders amount) fixed for all the network nodes.
2. Keeping the step size (value interval) fixed for all the network nodes regardless
of number of order they contain.
Defining a common bucket size for the networks results in arbitrary interval
boundaries for each node but it makes it possible to control their population.

Forcing production orders to take place in the nodes with constant interval
boundaries allows us to see how the aggregations take place within orders.
One should consider that the node populations might vary in large ranges.

Two alternative binning methods were performed to dense
feature networks. The first one, keeping the bucket size fixed is a model for the
Load/Count Driven Production type where the capacity of a campaign event is
a constraint for production. The second one, keeping the step size fixed is an
approach to mimic the Technology-Driven Production Type. In this production
type, sequences are derived based on some technological constraints which
may result in different ways for each feature. While we had no significant network
result for the former one, the Width feature network provided significant
graph communities with similar width values for the latter one as presented in
Fig. 3.5. That shows some orders within a specific width range in production
might be the consequence of a technical operation performed in the relevant
production unit.