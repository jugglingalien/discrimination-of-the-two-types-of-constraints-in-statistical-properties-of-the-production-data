Investigation of constraints impact in time windows was performed by analyze in two different type of association networks; the networks with fixed step size nodes and the networks with fixed bucket size nodes.

those two different type of networks were applied in all 10 time windows and some relevant network metric plots were generated. 
Average degree vs time windows
Average betweenness centrality vs time windows
Average modularity vs time windows

modularity equations\\
$Q = \frac {1} {4 m}\sum_ {ij} (A_{ij} - \frac {k_{i} k_{j}}{2 m}) \
s_{i} s_{j}$\\
$B_{ij} = A_{ij} - \frac {k_ {i} k_ {j}} {2 m}$\\
$Q = \frac {1} {4 m} s^{T} Bs = \frac {1} {4 m}\sum_ {i = 
	1}^{n} (u_ {i}^{T} . s)^{2}\beta_ {i}$\\
$\Delta Q = \frac {1} {4 m} s^{T} B^{(g)} s$\\
$B_{ij}^{(g)} = B_{ij} - \delta_{ij}\sum_ {k\in g} B_{ik}$

It is important to discuss how randomization has been done since the plot results can vary based on the generated null model via that specific randomization method. 
Random graphs were generated and below instruments were plotted.

Avrage degree vs time windows
Average betweenness centrality vs time windows
Modularity (Wolfram method) vs time windows
Modularity (GN method, algorithm by me) vs time windows
Average modularity for single random graph vs time windows
Z-scores for GN-modularity with Erdös-Renyi randomized null models vs time windows
Z-scores for GN-modularity randomized null models with fixed degree sequences vs time windows

Performing switch-randomization to a modular graph might fail even due to small details in randomization steps. That failure is probably the reason of high values of Z-scores in two different plots of Z-scores. If I try to switch randomize a modular graph, I could imagine a procedure where I take links only from same module and switch them or links that are across modules and switch them. And I mixed the sets of intra-module edges and inter-module edges separately. This null model might give a different result in Z-scores.

before treating time windows, having the modularity as function of bucket size and as function of step size. At this stage, choosing suitable step and bucket sizes and accordingly repeat all progress mentioned above. The aim is to obtain big amount of nodes as possible as we can and keeping that amount of nodes same in both graph structures (fixed step size and fixed bucket size). The difference between modularity values at highest graph nodes amount in fixed bucket and fixed step sized graphs shows which network structure is more effective on generating clear communities. In other words, modularity values is more meaningful when the node number is high.

