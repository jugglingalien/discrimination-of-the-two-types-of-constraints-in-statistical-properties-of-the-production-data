\chapter{Introduction}

\section{Background and Motivation}
{\color{red} 
	introducing the different types of constraints in manufacturing life cycle.
	
}
\clearpage
\section{Research Objective}

{\color{red} 
	
	A valid theoretical framework for discrimination of the two types of constraints in statistical properties of the production data
	
	First, formulate the binning methods here because this will describe the hypothesis underlining my thesis.
	
	Explanation of my hypothesis is a theoretical/conceptual framework as a starting point for the investigation. It is a well-defined object and based on facts.
	
	Our hypothesis: different types of constraints create non-random features in the association networks for different binning schemes. Network derived from other type (FBS?) of binning. they show non-random features mostly when I have performance constraints.
	
	Need to introduce different categories of constraints; technical constraints, performance-indicator based constraints in the context of FBA.
	
}

\section{Research Plan and Thesis Organization}

{\color{red} 
	
	Methods are introduced here as indicative of two fundamentally different constraints acting on the manufacturing process: technological constraints on the one hand and constraints related to material flow and production capacity on the other.
	
}