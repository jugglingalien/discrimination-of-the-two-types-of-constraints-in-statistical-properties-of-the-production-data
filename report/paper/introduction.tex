\chapter{Introduction}

\section{Background and Motivation}
{\color{red} 
	I should discuss and refer to different constraints from the literature introduced for the manufacturing life cycle.
	
	As a motivation, I need to introduce different categories of constraints as technical constraints, performance-indicator based constraints to be quantified in the context of the FBA in the further steps of this work. 
}
\clearpage
\section{Research Objective}

{\color{red} 
	Our hypothesis: different types of constraints create non-random features in the association networks for different binning schemes. Networks derived from various types of binning. Do they show non-random features when I have performance constraints or other types of constraints?
	
	Explanation of my hypothesis is a theoretical/conceptual framework as a starting point for the investigation. It is a well-defined valid object and based on facts. Moreover, it is structured to discriminate the two types of constraints in the statistical properties of the production data.
	
	The initial step of this master thesis work was to quantify the characteristics of two hypothetical types of constraints in industrial production: technology-driven constraints and load-driven constraints. 
}

\section{Research Plan and Thesis Organization}

{\color{red} 
	Methods are introduced here as indicative of two fundamentally different constraints acting on the manufacturing process: technological constraints on the one hand and constraints related to material flow and production capacity on the other.
	
	I plan to quantify the characteristics of two hypothetical types of constraints with an Operations Research Model consisting of two steps. First, analyzing the statistical properties of association networks over Time in an extensive data set from steel manufacturing; second, developing an abstract theoretical framework to understand better the connection between each type of constraint and the statistical patterns created by them. 
	
	Formulate the binning methods here because this will describe the hypothesis underlining my thesis.
	
	My Operations Research Model (OR model) combines Steel Manufacturing Events Analysis and Flux Balance Analysis. The art form of this model is to structure a standard data format and a shared analysis logic that allows comparing the results from manufacturing data and simulation data.
}